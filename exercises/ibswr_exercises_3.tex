% Options for packages loaded elsewhere
\PassOptionsToPackage{unicode}{hyperref}
\PassOptionsToPackage{hyphens}{url}
%
\documentclass[
]{article}
\usepackage{amsmath,amssymb}
\usepackage{iftex}
\ifPDFTeX
  \usepackage[T1]{fontenc}
  \usepackage[utf8]{inputenc}
  \usepackage{textcomp} % provide euro and other symbols
\else % if luatex or xetex
  \usepackage{unicode-math} % this also loads fontspec
  \defaultfontfeatures{Scale=MatchLowercase}
  \defaultfontfeatures[\rmfamily]{Ligatures=TeX,Scale=1}
\fi
\usepackage{lmodern}
\ifPDFTeX\else
  % xetex/luatex font selection
\fi
% Use upquote if available, for straight quotes in verbatim environments
\IfFileExists{upquote.sty}{\usepackage{upquote}}{}
\IfFileExists{microtype.sty}{% use microtype if available
  \usepackage[]{microtype}
  \UseMicrotypeSet[protrusion]{basicmath} % disable protrusion for tt fonts
}{}
\makeatletter
\@ifundefined{KOMAClassName}{% if non-KOMA class
  \IfFileExists{parskip.sty}{%
    \usepackage{parskip}
  }{% else
    \setlength{\parindent}{0pt}
    \setlength{\parskip}{6pt plus 2pt minus 1pt}}
}{% if KOMA class
  \KOMAoptions{parskip=half}}
\makeatother
\usepackage{xcolor}
\usepackage[margin=1in]{geometry}
\usepackage{color}
\usepackage{fancyvrb}
\newcommand{\VerbBar}{|}
\newcommand{\VERB}{\Verb[commandchars=\\\{\}]}
\DefineVerbatimEnvironment{Highlighting}{Verbatim}{commandchars=\\\{\}}
% Add ',fontsize=\small' for more characters per line
\usepackage{framed}
\definecolor{shadecolor}{RGB}{248,248,248}
\newenvironment{Shaded}{\begin{snugshade}}{\end{snugshade}}
\newcommand{\AlertTok}[1]{\textcolor[rgb]{0.94,0.16,0.16}{#1}}
\newcommand{\AnnotationTok}[1]{\textcolor[rgb]{0.56,0.35,0.01}{\textbf{\textit{#1}}}}
\newcommand{\AttributeTok}[1]{\textcolor[rgb]{0.13,0.29,0.53}{#1}}
\newcommand{\BaseNTok}[1]{\textcolor[rgb]{0.00,0.00,0.81}{#1}}
\newcommand{\BuiltInTok}[1]{#1}
\newcommand{\CharTok}[1]{\textcolor[rgb]{0.31,0.60,0.02}{#1}}
\newcommand{\CommentTok}[1]{\textcolor[rgb]{0.56,0.35,0.01}{\textit{#1}}}
\newcommand{\CommentVarTok}[1]{\textcolor[rgb]{0.56,0.35,0.01}{\textbf{\textit{#1}}}}
\newcommand{\ConstantTok}[1]{\textcolor[rgb]{0.56,0.35,0.01}{#1}}
\newcommand{\ControlFlowTok}[1]{\textcolor[rgb]{0.13,0.29,0.53}{\textbf{#1}}}
\newcommand{\DataTypeTok}[1]{\textcolor[rgb]{0.13,0.29,0.53}{#1}}
\newcommand{\DecValTok}[1]{\textcolor[rgb]{0.00,0.00,0.81}{#1}}
\newcommand{\DocumentationTok}[1]{\textcolor[rgb]{0.56,0.35,0.01}{\textbf{\textit{#1}}}}
\newcommand{\ErrorTok}[1]{\textcolor[rgb]{0.64,0.00,0.00}{\textbf{#1}}}
\newcommand{\ExtensionTok}[1]{#1}
\newcommand{\FloatTok}[1]{\textcolor[rgb]{0.00,0.00,0.81}{#1}}
\newcommand{\FunctionTok}[1]{\textcolor[rgb]{0.13,0.29,0.53}{\textbf{#1}}}
\newcommand{\ImportTok}[1]{#1}
\newcommand{\InformationTok}[1]{\textcolor[rgb]{0.56,0.35,0.01}{\textbf{\textit{#1}}}}
\newcommand{\KeywordTok}[1]{\textcolor[rgb]{0.13,0.29,0.53}{\textbf{#1}}}
\newcommand{\NormalTok}[1]{#1}
\newcommand{\OperatorTok}[1]{\textcolor[rgb]{0.81,0.36,0.00}{\textbf{#1}}}
\newcommand{\OtherTok}[1]{\textcolor[rgb]{0.56,0.35,0.01}{#1}}
\newcommand{\PreprocessorTok}[1]{\textcolor[rgb]{0.56,0.35,0.01}{\textit{#1}}}
\newcommand{\RegionMarkerTok}[1]{#1}
\newcommand{\SpecialCharTok}[1]{\textcolor[rgb]{0.81,0.36,0.00}{\textbf{#1}}}
\newcommand{\SpecialStringTok}[1]{\textcolor[rgb]{0.31,0.60,0.02}{#1}}
\newcommand{\StringTok}[1]{\textcolor[rgb]{0.31,0.60,0.02}{#1}}
\newcommand{\VariableTok}[1]{\textcolor[rgb]{0.00,0.00,0.00}{#1}}
\newcommand{\VerbatimStringTok}[1]{\textcolor[rgb]{0.31,0.60,0.02}{#1}}
\newcommand{\WarningTok}[1]{\textcolor[rgb]{0.56,0.35,0.01}{\textbf{\textit{#1}}}}
\usepackage{graphicx}
\makeatletter
\def\maxwidth{\ifdim\Gin@nat@width>\linewidth\linewidth\else\Gin@nat@width\fi}
\def\maxheight{\ifdim\Gin@nat@height>\textheight\textheight\else\Gin@nat@height\fi}
\makeatother
% Scale images if necessary, so that they will not overflow the page
% margins by default, and it is still possible to overwrite the defaults
% using explicit options in \includegraphics[width, height, ...]{}
\setkeys{Gin}{width=\maxwidth,height=\maxheight,keepaspectratio}
% Set default figure placement to htbp
\makeatletter
\def\fps@figure{htbp}
\makeatother
\setlength{\emergencystretch}{3em} % prevent overfull lines
\providecommand{\tightlist}{%
  \setlength{\itemsep}{0pt}\setlength{\parskip}{0pt}}
\setcounter{secnumdepth}{-\maxdimen} % remove section numbering
\usepackage{titling}
\renewcommand\maketitlehooka{%
  \setlength\parindent{0pt}%
  \vskip -5ex
  \begin{minipage}{\textwidth}
      \includegraphics[height=1cm]{./logos/cbg_logo.pdf}\hfill
      \includegraphics[height=0.6cm]{./logos/dbsse_logo.pdf}\hfill
      \includegraphics[height=0.6cm]{./logos/eth_logo.pdf}
  \end{minipage}\vskip 10ex
  \par
}
%\usepackage[table]{xcolor}
%\usepackage{tikz}
\usepackage{graphicx}
\usepackage{subfig}
%\usepackage{pgfplots}
%\usetikzlibrary{arrows}

\newcommand{\code}[1]{{\texttt{#1}}}
\ifLuaTeX
  \usepackage{selnolig}  % disable illegal ligatures
\fi
\usepackage{bookmark}
\IfFileExists{xurl.sty}{\usepackage{xurl}}{} % add URL line breaks if available
\urlstyle{same}
\hypersetup{
  pdftitle={Introduction to Bayesian Statistics with R},
  pdfauthor={Jack Kuipers},
  hidelinks,
  pdfcreator={LaTeX via pandoc}}

\title{Introduction to Bayesian Statistics with R}
\usepackage{etoolbox}
\makeatletter
\providecommand{\subtitle}[1]{% add subtitle to \maketitle
  \apptocmd{\@title}{\par {\large #1 \par}}{}{}
}
\makeatother
\subtitle{3: Exercises}
\author{Jack Kuipers}
\date{5 May 2025}

\begin{document}
\maketitle

\subsection{Exercise 3.1 - MCMC}\label{exercise-3.1---mcmc}

For MCMC we can walk randomly and accept according the the MH ratio to
eventually sample proportionally to any target distribution \(p(x)\)

\begin{Shaded}
\begin{Highlighting}[]
\CommentTok{\# simple MCMC function}
\CommentTok{\# n\_its is the number of iterations}
\CommentTok{\# start\_x the initial position}
\CommentTok{\# rw\_sd is the sd of the Gaussian random walk}
\NormalTok{basicMCMC }\OtherTok{\textless{}{-}} \ControlFlowTok{function}\NormalTok{(}\AttributeTok{n\_its =} \FloatTok{1e3}\NormalTok{, }\AttributeTok{start\_x =} \DecValTok{0}\NormalTok{, }\AttributeTok{rw\_sd =} \DecValTok{1}\NormalTok{, ...) \{}
\NormalTok{  xs }\OtherTok{\textless{}{-}} \FunctionTok{rep}\NormalTok{(}\ConstantTok{NA}\NormalTok{, n\_its) }\CommentTok{\# to store all the sampled values}
\NormalTok{  x }\OtherTok{\textless{}{-}}\NormalTok{ start\_x }\CommentTok{\# starting point}
\NormalTok{  xs[}\DecValTok{1}\NormalTok{] }\OtherTok{\textless{}{-}}\NormalTok{ x }\CommentTok{\# first value}
\NormalTok{  p\_x }\OtherTok{\textless{}{-}} \FunctionTok{target\_density}\NormalTok{(x, ...) }\CommentTok{\# probability density at current value of x}
  \ControlFlowTok{for}\NormalTok{ (ii }\ControlFlowTok{in} \DecValTok{2}\SpecialCharTok{:}\NormalTok{n\_its) \{ }\CommentTok{\# MCMC iterations}
\NormalTok{    x\_prop }\OtherTok{\textless{}{-}}\NormalTok{ x }\SpecialCharTok{+} \FunctionTok{rnorm}\NormalTok{(}\DecValTok{1}\NormalTok{, }\AttributeTok{mean =} \DecValTok{0}\NormalTok{, }\AttributeTok{sd =}\NormalTok{ rw\_sd) }\CommentTok{\# Gaussian random walk to propose next x}
\NormalTok{    p\_x\_prop }\OtherTok{\textless{}{-}} \FunctionTok{target\_density}\NormalTok{(x\_prop, ...) }\CommentTok{\# probability density at proposed x}
    \ControlFlowTok{if}\NormalTok{ (}\FunctionTok{runif}\NormalTok{(}\DecValTok{1}\NormalTok{) }\SpecialCharTok{\textless{}}\NormalTok{ p\_x\_prop}\SpecialCharTok{/}\NormalTok{p\_x) \{ }\CommentTok{\# MH acceptance probability}
\NormalTok{      x }\OtherTok{\textless{}{-}}\NormalTok{ x\_prop }\CommentTok{\# accept move}
\NormalTok{      p\_x }\OtherTok{\textless{}{-}}\NormalTok{ p\_x\_prop }\CommentTok{\# update density}
\NormalTok{    \}}
\NormalTok{    xs[ii] }\OtherTok{\textless{}{-}}\NormalTok{ x }\CommentTok{\# store current position, even when move rejected}
\NormalTok{  \}}
  \FunctionTok{return}\NormalTok{(xs)}
\NormalTok{\}}
\end{Highlighting}
\end{Shaded}

For example, if we want to sample from a Student-\(t\) distribution we
can use the following target

\begin{Shaded}
\begin{Highlighting}[]
\NormalTok{target\_density }\OtherTok{\textless{}{-}} \ControlFlowTok{function}\NormalTok{(x, nu) \{}
  \FunctionTok{dt}\NormalTok{(x, nu) }\CommentTok{\# Student{-}t density}
\NormalTok{\}}
\end{Highlighting}
\end{Shaded}

and run a short chain with \(\nu = 5\)

\begin{Shaded}
\begin{Highlighting}[]
\FunctionTok{basicMCMC}\NormalTok{(}\AttributeTok{nu =} \DecValTok{5}\NormalTok{)}
\FunctionTok{hist}\NormalTok{(}\FunctionTok{basicMCMC}\NormalTok{(}\AttributeTok{nu =} \DecValTok{5}\NormalTok{))}
\end{Highlighting}
\end{Shaded}

Examine the output MCMC chain for different lengths. How many samples
would we need to get close to the Student-\(t\) distribution?

Use the samples to estimate (see also the description in Bonus Exercise
3.3)

\[\int \cos(t) f_5(t) \mathrm{d} t \, , \qquad f_{\nu}(t) = \frac{\Gamma\left(\frac{\nu+1}{2}\right)}{\sqrt{\pi\nu}\Gamma\left(\frac{\nu}{2}\right)}\left(1+\frac{t^2}{\nu}\right)^{-\frac{\nu+1}{2}}\]
where \(f_{\nu}(t)\) is the probability density of a Student's
\(t\)-distribution with \(\nu\) degrees of freedom.

\subsection{Bonus Exercise 3.2 - HMC}\label{bonus-exercise-3.2---hmc}

\textbf{NOTE}: This exercise is an optional bonus for when you have
sufficient free time.

In HMC we move in the constant energy space (using Hamiltonian
mechanics) after we add another Gaussian dimension, rather than
randomly. With perfect propagation we would always accept the move,
allowing us to explore the space more efficiently. For computational
efficiency, we prefer to propagate numerically with a faster and more
approximate method (normally the leapfrog) and then accept according the
the MH ratio to eventually sample proportionally to any target
distribution \(p(x)\).

We define \(U(x) = -\log(p(x))\), and with a standard normal for the
extra dimension \(\rho\), we have the Hamiltonian

\[H(x, \rho) = U(x) + \frac{\rho^2}{2}\] while Hamilton's equations give
us the dynamics:

\[\frac{\mathrm{d}x}{\mathrm{d}t} = \frac{\partial H}{\partial \rho} = \rho \, , \quad   \frac{\mathrm{d}\rho}{\mathrm{d}t} = -\frac{\partial H}{\partial x} = - \frac{\mathrm{d}U}{\mathrm{d}x}\]

To propagate under these dynamics, we also need the gradient of the
target, so let's update our definition above for the Student-\(t\) while
keeping it back-compatible with before

\begin{Shaded}
\begin{Highlighting}[]
\NormalTok{target\_density }\OtherTok{\textless{}{-}} \ControlFlowTok{function}\NormalTok{(x, nu, }\AttributeTok{grad =} \ConstantTok{FALSE}\NormalTok{) \{}
\NormalTok{  dens }\OtherTok{\textless{}{-}} \FunctionTok{dt}\NormalTok{(x, nu) }\CommentTok{\# Student{-}t density}
  \ControlFlowTok{if}\NormalTok{ (grad) \{ }\CommentTok{\# return density and gradient}
\NormalTok{    grad }\OtherTok{\textless{}{-}} \SpecialCharTok{{-}}\NormalTok{(nu }\SpecialCharTok{+} \DecValTok{1}\NormalTok{)}\SpecialCharTok{/}\NormalTok{(nu }\SpecialCharTok{+}\NormalTok{ x}\SpecialCharTok{\^{}}\DecValTok{2}\NormalTok{)}\SpecialCharTok{*}\NormalTok{x}\SpecialCharTok{*}\NormalTok{dens}
    \FunctionTok{return}\NormalTok{(}\FunctionTok{list}\NormalTok{(}\AttributeTok{dens =}\NormalTok{ dens, }\AttributeTok{grad =}\NormalTok{ grad))}
\NormalTok{  \} }\ControlFlowTok{else}\NormalTok{ \{ }\CommentTok{\# return just the density}
    \FunctionTok{return}\NormalTok{(dens)}
\NormalTok{  \}}
\NormalTok{\}}
\end{Highlighting}
\end{Shaded}

and define our function \(U\)

\begin{Shaded}
\begin{Highlighting}[]
\NormalTok{U\_fn }\OtherTok{\textless{}{-}} \ControlFlowTok{function}\NormalTok{(x, ...) \{}
\NormalTok{  p\_x }\OtherTok{\textless{}{-}} \FunctionTok{target\_density}\NormalTok{(x, ..., }\AttributeTok{grad =} \ConstantTok{TRUE}\NormalTok{)}
\NormalTok{  U }\OtherTok{\textless{}{-}} \SpecialCharTok{{-}}\FunctionTok{log}\NormalTok{(p\_x}\SpecialCharTok{$}\NormalTok{dens)}
\NormalTok{  grad }\OtherTok{\textless{}{-}} \SpecialCharTok{{-}}\DecValTok{1}\SpecialCharTok{/}\NormalTok{p\_x}\SpecialCharTok{$}\NormalTok{dens}\SpecialCharTok{*}\NormalTok{p\_x}\SpecialCharTok{$}\NormalTok{grad}
  \FunctionTok{return}\NormalTok{(}\FunctionTok{list}\NormalTok{(}\AttributeTok{U =}\NormalTok{ U, }\AttributeTok{grad =}\NormalTok{ grad))}
\NormalTok{\}}
\end{Highlighting}
\end{Shaded}

Now we're ready for our HMC code. This is similar to the MCMC code, with
an internal loop for the leapfrog propagation. Since this takes \(L\)
steps, we shorten the number of outside iterations a bit to compensate.

\begin{Shaded}
\begin{Highlighting}[]
\CommentTok{\# simple HMC function}
\CommentTok{\# n\_its is the number of iterations}
\CommentTok{\# start\_x the initial position}
\CommentTok{\# L is the number of steps of numerical propagation}
\CommentTok{\# under the Hamiltionian H = U + rho\^{}2/2, U = {-}log(target\_density)}
\CommentTok{\# epsilon is the size of the steps}
\NormalTok{basicHMC }\OtherTok{\textless{}{-}} \ControlFlowTok{function}\NormalTok{(}\AttributeTok{n\_its =} \FloatTok{1e2}\NormalTok{, }\AttributeTok{start\_x =} \DecValTok{0}\NormalTok{, }\AttributeTok{L =} \DecValTok{10}\NormalTok{, }\AttributeTok{epsilon =} \FloatTok{0.1}\NormalTok{, ...) \{}
\NormalTok{  xs }\OtherTok{\textless{}{-}} \FunctionTok{rep}\NormalTok{(}\ConstantTok{NA}\NormalTok{, n\_its) }\CommentTok{\# to store all the sampled values}
\NormalTok{  x }\OtherTok{\textless{}{-}}\NormalTok{ start\_x }\CommentTok{\# starting point}
\NormalTok{  xs[}\DecValTok{1}\NormalTok{] }\OtherTok{\textless{}{-}}\NormalTok{ x }\CommentTok{\# first value}
\NormalTok{  U\_x }\OtherTok{\textless{}{-}} \FunctionTok{U\_fn}\NormalTok{(x, ...) }\CommentTok{\# log density and gradient at current x}
  \ControlFlowTok{for}\NormalTok{ (ii }\ControlFlowTok{in} \DecValTok{2}\SpecialCharTok{:}\NormalTok{n\_its) \{ }\CommentTok{\# HMC iterations}
\NormalTok{    rho }\OtherTok{\textless{}{-}} \FunctionTok{rnorm}\NormalTok{(}\DecValTok{1}\NormalTok{) }\CommentTok{\# normal sample (we could define scheme with different sd)}
\NormalTok{    x\_prop }\OtherTok{\textless{}{-}}\NormalTok{ x }
    \CommentTok{\# Leapfrog method to propagate under Hamiltonian: }
\NormalTok{    rho\_prop }\OtherTok{\textless{}{-}}\NormalTok{ rho }\SpecialCharTok{{-}}\NormalTok{ epsilon}\SpecialCharTok{/}\DecValTok{2}\SpecialCharTok{*}\NormalTok{U\_x}\SpecialCharTok{$}\NormalTok{grad }\CommentTok{\# half step for momentum}
    \ControlFlowTok{for}\NormalTok{ (j }\ControlFlowTok{in} \DecValTok{1}\SpecialCharTok{:}\NormalTok{L) \{}
\NormalTok{      x\_prop }\OtherTok{\textless{}{-}}\NormalTok{ x\_prop }\SpecialCharTok{+}\NormalTok{ epsilon}\SpecialCharTok{*}\NormalTok{rho\_prop }\CommentTok{\# position update }
\NormalTok{      U\_prop }\OtherTok{\textless{}{-}} \FunctionTok{U\_fn}\NormalTok{(x\_prop, ...) }\CommentTok{\# update gradient}
      \CommentTok{\# update momentum, with a half step at the end}
\NormalTok{      rho\_prop }\OtherTok{\textless{}{-}}\NormalTok{ rho\_prop }\SpecialCharTok{{-}}\NormalTok{ epsilon}\SpecialCharTok{*}\NormalTok{U\_prop}\SpecialCharTok{$}\NormalTok{grad}\SpecialCharTok{/}\NormalTok{(}\DecValTok{1} \SpecialCharTok{+}\NormalTok{ (j}\SpecialCharTok{==}\NormalTok{L))}
\NormalTok{    \}}
\NormalTok{    MH\_prob }\OtherTok{\textless{}{-}} \FunctionTok{exp}\NormalTok{(U\_x}\SpecialCharTok{$}\NormalTok{U }\SpecialCharTok{+}\NormalTok{ rho}\SpecialCharTok{\^{}}\DecValTok{2}\SpecialCharTok{/}\DecValTok{2} \SpecialCharTok{{-}}\NormalTok{ U\_prop}\SpecialCharTok{$}\NormalTok{U }\SpecialCharTok{{-}}\NormalTok{ rho\_prop}\SpecialCharTok{\^{}}\DecValTok{2}\SpecialCharTok{/}\DecValTok{2}\NormalTok{)}
    \ControlFlowTok{if}\NormalTok{ (}\FunctionTok{runif}\NormalTok{(}\DecValTok{1}\NormalTok{) }\SpecialCharTok{\textless{}}\NormalTok{ MH\_prob) \{ }\CommentTok{\# MH acceptance probability}
\NormalTok{      x }\OtherTok{\textless{}{-}}\NormalTok{ x\_prop }\CommentTok{\# accept move}
\NormalTok{      U\_x }\OtherTok{\textless{}{-}}\NormalTok{ U\_prop }\CommentTok{\# update density}
\NormalTok{    \}}
\NormalTok{    xs[ii] }\OtherTok{\textless{}{-}}\NormalTok{ x }\CommentTok{\# store current position, even when move rejected}
\NormalTok{  \}}
  \FunctionTok{return}\NormalTok{(xs)}
\NormalTok{\}}
\end{Highlighting}
\end{Shaded}

Now we can run a short chain again with \(\nu = 5\)

\begin{Shaded}
\begin{Highlighting}[]
\FunctionTok{basicHMC}\NormalTok{(}\AttributeTok{nu =} \DecValTok{5}\NormalTok{)}
\end{Highlighting}
\end{Shaded}

and examine the output HMC chain for different lengths. How many samples
do we now need to get close to the Student-\(t\) distribution?

Do we get good estimates for the integral from before?

\subsection{Bonus Exercise 3.3 - Monte Carlo
integration}\label{bonus-exercise-3.3---monte-carlo-integration}

\textbf{NOTE}: This exercise is an optional bonus for when you have
sufficient free time.

Computing expectations can be applied to any continuous function

\[E[g(x)] = \int g(x) p(x) \mathrm{d} x\]

so that integrals where we recognise \(p(x)\) as (proportional to) a
probability distribution may be estimated with Monte Carlo methods since

\[E[g(x)] \approx \frac{1}{M} \sum_{i=1}^{M} g(x_i) \]

for \(M\) random samples \(x_i\) sampled according to \(p(x)\). Use
samples from a Gaussian to estimate the following three integrals:

\[\int \vert x \vert \mathrm{e}^{-x^2} \mathrm{d} x \, , \qquad \int \sin(x) \mathrm{e}^{-x^2} \mathrm{d} x\, , \qquad \int \cos(x) \mathrm{e}^{-x^2} \mathrm{d} x\]

\textbf{Reminder}, the Gaussian probability density has the following
general form:

\[
p(x) = \frac{1}{\sqrt{2\pi\sigma^2}} \mathrm{e}^{- \frac{(x - \mu)^ 2}{ 2\sigma^2} }
\]

Compare the estimated values to the exact values of the integrals.

\textbf{NOTE:} For the comparison to the real values, you can integrate
analytically or use \texttt{R}'s \texttt{integrate} function. The errors
from the true value, like standard errors in general, decrease like
\(\frac{1}{\sqrt{M}}\).

\end{document}
